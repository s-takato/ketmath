%%% タイトル presen22204
\documentclass[landscape,10pt]{jarticle}
\special{papersize=\the\paperwidth,\the\paperheight}
\usepackage{ketpic,ketlayer}
\usepackage{ketslide}
\usepackage{amsmath,amssymb}
\usepackage{bm,enumerate}
\usepackage[dvipdfmx]{graphicx}
\usepackage{color}
\definecolor{slidecolora}{cmyk}{0.98,0.13,0,0.43}
\definecolor{slidecolorb}{cmyk}{0.2,0,0,0}
\definecolor{slidecolorc}{cmyk}{0.2,0,0,0}
\definecolor{slidecolord}{cmyk}{0.2,0,0,0}
\definecolor{slidecolore}{cmyk}{0,0,0,0.5}
\definecolor{slidecolorf}{cmyk}{0,0,0,0.5}
\definecolor{slidecolori}{cmyk}{0.98,0.13,0,0.43}
\def\setthin#1{\def\thin{#1}}
\setthin{0}
\newcommand{\slidepage}[1][s]{%
\setcounter{ketpicctra}{18}%
\if#1m \setcounter{ketpicctra}{1}\fi
\hypersetup{linkcolor=black}%

\begin{layer}{118}{0}
\putnotee{122}{-\theketpicctra.05}{\small\thepage/\pageref{pageend}}
\end{layer}\hypersetup{linkcolor=blue}

}
\usepackage{emath}
\usepackage{pict2e}
\usepackage{ketlayermorewith2e}
\usepackage[dvipdfmx,colorlinks=true,linkcolor=blue,filecolor=blue]{hyperref}
\newcommand{\hiduke}{0829}
\newcommand{\hako}[2][1]{\fbox{\raisebox{#1mm}{\mbox{}}\raisebox{-#1mm}{\mbox{}}\,\phantom{#2}\,}}
\newcommand{\hakoa}[2][1]{\fbox{\raisebox{#1mm}{\mbox{}}\raisebox{-#1mm}{\mbox{}}\,#2\,}}
\newcommand{\hakom}[2][1]{\hako[#1]{$#2$}}
\newcommand{\hakoma}[2][1]{\hakoa[#1]{$#2$}}
\def\rad{\;\mathrm{rad}}
\def\deg#1{#1^{\circ}}
\newcommand{\sbunsuu}[2]{\scalebox{0.6}{$\bunsuu{#1}{#2}$}}
\def\pow{$\hspace{-1.5mm}^\hspace{-1mm}$}
\def\dlim{\displaystyle\lim}
\newcommand{\brd}[2][1]{\scalebox{#1}{\color{red}\fbox{\color{black}$#2$}}}
\newcommand\down[1][0.5zw]{\vspace{#1}\\}
\newcommand{\sfrac}[3][0.65]{\scalebox{#1}{$\frac{#2}{#3}$}}
\newcommand{\phn}[1]{\phantom{#1}}
\newcommand{\scb}[2][0.6]{\scalebox{#1}{#2}}
\newcommand{\dsum}{\displaystyle\sum}

\setmargin{25}{145}{15}{100}

\ketslideinit

\pagestyle{empty}

\begin{document}

\begin{layer}{120}{0}
\putnotese{0}{0}{\input{fig/slide0829.tex}}
\end{layer}

\def\mainslidetitley{22}
\def\ketcletter{slidecolora}
\def\ketcbox{slidecolorb}
\def\ketdbox{slidecolorc}
\def\ketcframe{slidecolord}
\def\ketcshadow{slidecolore}
\def\ketdshadow{slidecolorf}
\def\slidetitlex{6}
\def\slidetitlesize{1.3}
\def\mketcletter{slidecolori}
\def\mketcbox{yellow}
\def\mketdbox{yellow}
\def\mketcframe{yellow}
\def\mslidetitlex{62}
\def\mslidetitlesize{2}

\color{black}
\Large\bf\boldmath
\addtocounter{page}{-1}

\def\MARU{}
\renewcommand{\MARU}[1]{{\ooalign{\hfil$#1$\/\hfil\crcr\raise.167ex\hbox{\mathhexbox20D}}}}
\renewcommand{\slidepage}[1][s]{%
\setcounter{ketpicctra}{18}%
\if#1m \setcounter{ketpicctra}{1}\fi
\hypersetup{linkcolor=black}%
\begin{layer}{118}{0}
\putnotee{115}{-\theketpicctra.05}{\small\hiduke-\thepage/\pageref{pageend}}
\end{layer}\hypersetup{linkcolor=blue}
}
\newcounter{ban}
\setcounter{ban}{1}
\newcommand{\monban}[1][\hiduke]{%
#1-\theban\ %
\addtocounter{ban}{1}%
}
\newcommand{\monbannoadd}[1][\hiduke]{%
#1-\theban\ %
}
\newcommand{\addban}{%
\addtocounter{ban}{1}%
}
\newcounter{edawidth}
\newcounter{edactr}
\newcommand{\seteda}[1]{% 20220708 modified
\setcounter{edawidth}{#1}
\setcounter{edactr}{1}
}
\newcommand{\eda}[2][\theedawidth]{%
\Ltab{#1 mm}{[\theedactr]\ #2}%
\addtocounter{edactr}{1}%
}
%%%%%%%%%%%%

%%%%%%%%%%%%%%%%%%%%

\mainslide{ 復習(微分)}


\slidepage[m]
%%%%%%%%%%%%%

%%%%%%%%%%%%%%%%%%%%

\newslide{微分と導関数}

\vspace*{18mm}

\slidepage
\begin{itemize}
\item
$a$における微分係数$f'(a)=$接線の傾き\\
\hspace*{2zw}$f'(a)=\dlim_{z\to a}\bunsuu{f(z)-f(a)}{z-a}$\vspace{-2mm}
\end{itemize}
%%%%%%%%%%%%%

%%%%%%%%%%%%%%%%%%%%


\sameslide

\vspace*{18mm}

\slidepage
\begin{itemize}
\item
$a$における微分係数$f'(a)=$接線の傾き\\
\hspace*{2zw}$f'(a)=\dlim_{z\to a}\bunsuu{f(z)-f(a)}{z-a}$\vspace{-2mm}
\item
導関数\\
・微分係数を$x$の関数$f'(x)$としたもの\\
\hspace*{2zw}$y'=f'(x)=\dlim_{z\to x}\bunsuu{f(z)-f(x)}{z-x}$\\
・導関数を求めることを「微分する」
\end{itemize}

\sameslide

\vspace*{18mm}

\slidepage
\begin{itemize}
\item
$a$における微分係数$f'(a)=$接線の傾き\\
\hspace*{2zw}$f'(a)=\dlim_{z\to a}\bunsuu{f(z)-f(a)}{z-a}$\vspace{-2mm}
\item
導関数\\
・微分係数を$x$の関数$f'(x)$としたもの\\
\hspace*{2zw}$y'=f'(x)=\dlim_{z\to x}\bunsuu{f(z)-f(x)}{z-x}$\\
・導関数を求めることを「微分する」
\item
[課題]\monban 次の関数の導関数はどうなるか.\seteda{50}\\
\eda{$y=x^2-2x$}\eda{$y=\sin x$}
\end{itemize}

\newslide{$x^n$の微分}

\vspace*{18mm}

\slidepage
\begin{itemize}
\item
$(c)'=0$($c$は定数)
\item
$(x)'=1$
\item
$(x^2)'=2x$
\item
$(x^3)'=3x^2$
\item
一般に $(x^n)'=n x^{n-1}$\\
 ・$n$は負の整数でも分数(実数)でもよい
\end{itemize}
%%%%%%%%%%%%%

%%%%%%%%%%%%%%%%%%%%


\newslide{三角関数の微分}

\vspace*{18mm}

\slidepage
\begin{itemize}
\item
$y=\sin x, \cos x, \tan x$
\end{itemize}
%%%%%%%%%%%%

%%%%%%%%%%%%%%%%%%%%


\sameslide

\vspace*{18mm}

\slidepage
\begin{itemize}
\item
$y=\sin x, \cos x, \tan x$
\item
角$x$の単位はラジアン
\end{itemize}

\sameslide

\vspace*{18mm}

\slidepage
\begin{itemize}
\item
$y=\sin x, \cos x, \tan x$
\item
角$x$の単位はラジアン
\item
$(\sin x)'=\cos x$
\item
$(\cos x)'=-\sin x$
\item
$(\tan x)'=\bunsuu{1}{\cos^2 x}$
\end{itemize}

\sameslide

\vspace*{18mm}

\slidepage
\begin{itemize}
\item
$y=\sin x, \cos x, \tan x$
\item
角$x$の単位はラジアン
\item
$(\sin x)'=\cos x$
\item
$(\cos x)'=-\sin x$
\item
$(\tan x)'=\bunsuu{1}{\cos^2 x}$
\item
[課題]\monban 次の関数を微分せよ.\seteda{55}\\
\eda{$y=x+\cos x$}\eda{$y=x\sin x$}
\end{itemize}

\newslide{指数関数$y=e^{x}$の微分}

\vspace*{18mm}

\slidepage
\begin{itemize}
\item
$e$はネピアの定数
\end{itemize}
%%%%%%%%%%%%

%%%%%%%%%%%%%%%%%%%%


\sameslide

\vspace*{18mm}

\slidepage
\begin{itemize}
\item
$e$はネピアの定数
\item
[課題]\monbannoadd $e$の値を小数点以下5位まで書け.
\end{itemize}

\sameslide

\vspace*{18mm}

\slidepage
\begin{itemize}
\item
$e$はネピアの定数
\item
[課題]\monbannoadd $e$の値を小数点以下5位まで書け.
\item
\fbox{$(e^x)'=e^x$}
\end{itemize}

\sameslide

\vspace*{18mm}

\slidepage
\begin{itemize}
\item
$e$はネピアの定数
\item
[課題]\monbannoadd $e$の値を小数点以下5位まで書け.
\addban
\item
\fbox{$(e^x)'=e^x$}
\item
[課題]\monban 次の関数を微分せよ.\seteda{40}\\
\eda[46]{$y=e^x+x^2$}\eda{$y=e^{2x}$}\eda{$y=e^{-x}$}
\end{itemize}

\newslide{自然対数$y=\log x(=\ln x)$の微分}

\vspace*{18mm}

\hypertarget{para1pg1}{}

\begin{layer}{120}{0}
\putnotew{96}{73}{\hyperlink{para0pg4}{\fbox{\Ctab{2.5mm}{\scalebox{1}{\scriptsize $\mathstrut||\!\lhd$}}}}}
\putnotew{101}{73}{\hyperlink{para1pg1}{\fbox{\Ctab{2.5mm}{\scalebox{1}{\scriptsize $\mathstrut|\!\lhd$}}}}}
\putnotew{108}{73}{\hyperlink{para0pg4}{\fbox{\Ctab{4.5mm}{\scalebox{1}{\scriptsize $\mathstrut\!\!\lhd\!\!$}}}}}
\putnotew{115}{73}{\hyperlink{para1pg2}{\fbox{\Ctab{4.5mm}{\scalebox{1}{\scriptsize $\mathstrut\!\rhd\!$}}}}}
\putnotew{120}{73}{\hyperlink{para1pg3}{\fbox{\Ctab{2.5mm}{\scalebox{1}{\scriptsize $\mathstrut \!\rhd\!\!|$}}}}}
\putnotew{125}{73}{\hyperlink{para2pg1}{\fbox{\Ctab{2.5mm}{\scalebox{1}{\scriptsize $\mathstrut \!\rhd\!\!||$}}}}}
\putnotee{126}{73}{\scriptsize\color{blue} 1/3}
\end{layer}

\slidepage
\begin{itemize}
\item
ネイピア数$e$を底とする対数
\item
[]\hspace*{2zw}$y=\log x\ \Longleftrightarrow\ e^y=x$
\end{itemize}

\sameslide

\vspace*{18mm}

\hypertarget{para1pg2}{}

\begin{layer}{120}{0}
\putnotew{96}{73}{\hyperlink{para0pg4}{\fbox{\Ctab{2.5mm}{\scalebox{1}{\scriptsize $\mathstrut||\!\lhd$}}}}}
\putnotew{101}{73}{\hyperlink{para1pg1}{\fbox{\Ctab{2.5mm}{\scalebox{1}{\scriptsize $\mathstrut|\!\lhd$}}}}}
\putnotew{108}{73}{\hyperlink{para1pg1}{\fbox{\Ctab{4.5mm}{\scalebox{1}{\scriptsize $\mathstrut\!\!\lhd\!\!$}}}}}
\putnotew{115}{73}{\hyperlink{para1pg3}{\fbox{\Ctab{4.5mm}{\scalebox{1}{\scriptsize $\mathstrut\!\rhd\!$}}}}}
\putnotew{120}{73}{\hyperlink{para1pg3}{\fbox{\Ctab{2.5mm}{\scalebox{1}{\scriptsize $\mathstrut \!\rhd\!\!|$}}}}}
\putnotew{125}{73}{\hyperlink{para2pg1}{\fbox{\Ctab{2.5mm}{\scalebox{1}{\scriptsize $\mathstrut \!\rhd\!\!||$}}}}}
\putnotee{126}{73}{\scriptsize\color{blue} 2/3}
\end{layer}

\slidepage
\begin{itemize}
\item
ネイピア数$e$を底とする対数
\item
[]\hspace*{2zw}$y=\log x\ \Longleftrightarrow\ e^y=x$
\item
\fbox{$(\log x)'=\bunsuu{1}{x}$}
\end{itemize}

\sameslide

\vspace*{18mm}

\hypertarget{para1pg3}{}

\begin{layer}{120}{0}
\putnotew{96}{73}{\hyperlink{para0pg4}{\fbox{\Ctab{2.5mm}{\scalebox{1}{\scriptsize $\mathstrut||\!\lhd$}}}}}
\putnotew{101}{73}{\hyperlink{para1pg1}{\fbox{\Ctab{2.5mm}{\scalebox{1}{\scriptsize $\mathstrut|\!\lhd$}}}}}
\putnotew{108}{73}{\hyperlink{para1pg2}{\fbox{\Ctab{4.5mm}{\scalebox{1}{\scriptsize $\mathstrut\!\!\lhd\!\!$}}}}}
\putnotew{115}{73}{\hyperlink{para1pg3}{\fbox{\Ctab{4.5mm}{\scalebox{1}{\scriptsize $\mathstrut\!\rhd\!$}}}}}
\putnotew{120}{73}{\hyperlink{para1pg3}{\fbox{\Ctab{2.5mm}{\scalebox{1}{\scriptsize $\mathstrut \!\rhd\!\!|$}}}}}
\putnotew{125}{73}{\hyperlink{para2pg1}{\fbox{\Ctab{2.5mm}{\scalebox{1}{\scriptsize $\mathstrut \!\rhd\!\!||$}}}}}
\putnotee{126}{73}{\scriptsize\color{blue} 3/3}
\end{layer}

\slidepage
\begin{itemize}
\item
ネイピア数$e$を底とする対数
\item
[]\hspace*{2zw}$y=\log x\ \Longleftrightarrow\ e^y=x$
\item
\fbox{$(\log x)'=\bunsuu{1}{x}$}
\item
[課題]\monban 次の関数を微分せよ.\seteda{55}\\
\eda{$y=\log x+e^{x}$}\eda{$y=\log 2x$}
\end{itemize}
\label{pageend}\mbox{}

\end{document}
